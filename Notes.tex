% boiler-plate base document generates by Claude Sonnet 4
\documentclass[12pt,a4paper]{article}

% Essential packages for XeLaTeX
\usepackage{fontspec}
\usepackage{xunicode}
\usepackage{xltxtra}

% Set main font (using more universally available fonts)
% Option 1: Use TeX Gyre fonts (included with most LaTeX distributions)
\setmainfont{TeX Gyre Termes}  % Times-like font
\setsansfont{TeX Gyre Heros}   % Helvetica-like font
\setmonofont{TeX Gyre Cursor}  % Courier-like font

% Option 2: Alternative fonts if TeX Gyre isn't available (uncomment to use)
% \setmainfont{Liberation Serif}
% \setsansfont{Liberation Sans}
% \setmonofont{Liberation Mono}

% Option 3: Fallback to basic fonts (uncomment if others don't work)
% \setmainfont{Latin Modern Roman}
% \setsansfont{Latin Modern Sans}
% \setmonofont{Latin Modern Mono}

% Language and encoding
\usepackage{polyglossia}
\setdefaultlanguage{english}

% Mathematics
\usepackage{amsmath}
\usepackage{amsfonts}
\usepackage{amssymb}
\usepackage{mathtools}

% Graphics and figures
\usepackage{graphicx}
\usepackage{float}
\usepackage{subcaption}

% Tables
\usepackage{booktabs}
\usepackage{array}

% Hyperlinks and cross-references
\usepackage[colorlinks=true,
            linkcolor=blue,
            urlcolor=blue,
            citecolor=red,
            filecolor=magenta]{hyperref}
\usepackage{cleveref} % For smart cross-referencing

% Bibliography
\usepackage[style=authoryear,backend=biber]{biblatex}
\addbibresource{references.bib}

% Page layout
\usepackage[margin=1in]{geometry}
\usepackage{setspace}
\onehalfspacing

% Headers and footers
\usepackage{fancyhdr}
\pagestyle{fancy}
\fancyhf{}
\rhead{\thepage}
\lhead{\leftmark}

% Additional useful packages
\usepackage{enumitem}
\usepackage{tikz} % For drawing diagrams
\usepackage{listings} % For code blocks
\usepackage{xcolor}

% Title page information
\title{Notes and Thoughts on Implementing Windowed-Multipole through Vector Fitting}
\author{Andrew Holcomb}
\date{\today}

% Custom commands (optional)
\newcommand{\FIXME}[1]{\textcolor{red}{\textbf{#1}}}

\begin{document}

\maketitle

\tableofcontents
\newpage

\section{Introduction}\label{sec:introduction}

This document demonstrates various LaTeX features including cross-references, hyperlinks, equations, figures, and citations. This is a sample introduction that references \Cref{sec:equations} and \Cref{fig:sample-image}.

You can create hyperlinks to external websites like \href{https://www.latex-project.org/}{The LaTeX Project} or \href{https://www.overleaf.com/}{Overleaf}.

\section{Mathematical Equations}\label{sec:equations}

Here are some examples of numbered equations that will be automatically numbered:

\begin{equation}\label{eq:quadratic}
ax^2 + bx + c = 0
\end{equation}

The solutions to \Cref{eq:quadratic} are given by the quadratic formula:

\begin{equation}\label{eq:quadratic-formula}
x = \frac{-b \pm \sqrt{b^2 - 4ac}}{2a}
\end{equation}

You can also have multi-line equations:

\begin{align}\label{eq:multi-line}
f(x) &= x^2 + 2x + 1 \\
&= (x + 1)^2 \label{eq:factored}
\end{align}

As shown in \Cref{eq:factored}, the expression can be factored.

For inline math, you can write $E = mc^2$ or $\int_{-\infty}^{\infty} e^{-x^2} dx = \sqrt{\pi}$.

\section{Figures and Images}\label{sec:figures}

\begin{figure}[H]
    \centering
    \includegraphics[width=\textwidth]{o16-2-u235-18.png}

    % For demonstration, using a TikZ drawing since we don't have an actual image
    %\begin{tikzpicture}
    %    \draw[thick] (0,0) rectangle (4,3);
    %    \draw[thick] (0,0) -- (4,3);
    %    \draw[thick] (0,3) -- (4,0);
    %    \node at (2,1.5) {\Large Sample Figure};
    %\end{tikzpicture}

    \caption{This is a sample figure demonstrating figure inclusion.}
    \label{fig:sample-image}
\end{figure}

As you can see in \Cref{fig:sample-image}, figures are automatically numbered and can be cross-referenced throughout the document.

\subsection{Multiple Subfigures}

You can also create subfigures:

\begin{figure}[H]
    \centering
    \begin{subfigure}{0.45\textwidth}
        \centering
        \begin{tikzpicture}
            \draw[fill=blue!20] (0,0) circle (1);
            \node at (0,0) {A};
        \end{tikzpicture}
        \caption{First subfigure}
        \label{fig:sub1}
    \end{subfigure}
    \hfill
    \begin{subfigure}{0.45\textwidth}
        \centering
        \begin{tikzpicture}
            \draw[fill=red!20] (0,0) rectangle (2,1.5);
            \node at (1,0.75) {B};
        \end{tikzpicture}
        \caption{Second subfigure}
        \label{fig:sub2}
    \end{subfigure}
    \caption{Example of subfigures with \protect\subref{fig:sub1} showing a circle and \protect\subref{fig:sub2} showing a rectangle.}
    \label{fig:subfigures}
\end{figure}

\section{Tables}\label{sec:tables}

\begin{table}[H]
    \centering
    \caption{Sample data table}
    \label{tab:sample-data}
    \begin{tabular}{@{}lcc@{}}
        \toprule
        Parameter & Value & Unit \\
        \midrule
        Temperature & 25.4 & °C \\
        Pressure & 101.3 & kPa \\
        Humidity & 65 & \% \\
        \bottomrule
    \end{tabular}
\end{table}

The data in \Cref{tab:sample-data} shows typical environmental conditions.

\section{Citations and References}\label{sec:citations}

You can cite references using various styles. For example, you might reference a seminal paper \parencite{einstein1905} or discuss the work of \textcite{newton1687}.

Multiple citations can be grouped together \parencite{einstein1905,newton1687}.

\section{Cross-References and Hyperlinks}\label{sec:cross-refs}

This document demonstrates several types of cross-references:

\begin{itemize}
    \item Section references: \Cref{sec:introduction}, \Cref{sec:equations}
    \item Equation references: \Cref{eq:quadratic}, \Cref{eq:quadratic-formula}
    \item Figure references: \Cref{fig:sample-image}, \Cref{fig:subfigures}
    \item Table references: \Cref{tab:sample-data}
\end{itemize}

External links include:
\begin{itemize}
    \item \href{https://www.ctan.org/}{CTAN - The Comprehensive TeX Archive Network}
    \item \href{https://tex.stackexchange.com/}{TeX Stack Exchange}
    \item \href{https://detexify.kirelabs.org/classify.html}{Detexify - LaTeX symbol recognition}
\end{itemize}

\section{Code and Listings}

You can include code snippets:

\begin{lstlisting}[language=Python, caption=Sample Python code, label=lst:python]
def hello_world():
    print("Hello, World!")
    return True

if __name__ == "__main__":
    hello_world()
\end{lstlisting}

\section{Conclusion}

This boilerplate document provides a solid foundation for academic and technical writing with LaTeX. It includes all the essential packages and examples for the features you requested.

\newpage
\printbibliography

\end{document}
